\section{Uvod}

U hrvatskom se obrazovanju posljednjih dvadesetak godina jasno potvrđuje snažan
rast „sjene obrazovanja” {--} razgranat obrazovni biznis plaćenih privatnih
instrukcija i organiziranih priprema za državnu maturu i prijemne ispite, koji
funkcionira paralelno s formalnim obrazovnim sustavom. Dodatne poduke postale
su uobičajen dio školovanja i svakodnevica mnogih hrvatskih đaka svih uzrasta
    {--} od osnovne škole do fakulteta, a ne iznimka, unatoč tome što ovakav oblik
obrazovanja otvara niz pitanja jednakosti pristupa znanju (navedene usluge
dostupnije su učenicima iz socioekonomski povoljnijih obitelji te iz većih
geografskih centara) te stvara pritisak na učenike i komercijalizira
obrazovanje. Istraživanja Borisa Jokića i Zrinke Ristić Dedić pokazuju da je
gotovo 40,0\% učenika u svakoj ispitanoj generaciji (8.\ razred osnovne te 2.\
i završni razred srednje škole) koristilo privatne instrukcije u školskoj
godini 2020./2021., a među maturantima ih je 38,0\%. Svaki drugi maturant
gimnazije pohađa instrukcije (i to 17,1\% njih redovito), dok 39,2\% učenika
završnih razreda strukovnih i umjetničkih škola također koristi takvu pomoć,
najčešće iz matematike i ključnih predmeta za upis na studij. Većina učenika i
dalje preferira pohađanje repeticija uživo, međutim 20,6\% maturanata i 15,9\%
osmaša pohađalo je privatne instrukcije u šk.\ god.\ 2020./2021.\ u \textit{online}
obliku~\cite{Jokic-RisticDedic-2022-PrivateTutoring}. 56,0\% maturanata
pohađalo je pripremne tečajeve za državnu maturu, a njih 36,1\% pohađalo ih je
u nekom obliku kod privatnih tvrtki. 62,9\% gimnazijalaca i 50,3\% učenika
strukovnih škola koji planiraju upisati studij pohađalo je pripremne tečajeve u
šk.\ god.\ 2020./2021., pri čemu 48,6\% gimnazijalaca i 26,4\%
\textit{strukovnjaka} kod privatnih
tvrtki~\cite{Jokic-RisticDedic-2022-PreparatoryCourses}.

Trenutno se ponuda privatnih repeticija u Hrvatskoj najčešće pronalazi putem
usmenih preporuka, \textit{online} oglasnika ili grupa na društvenim mrežama, kao i na
zalijepljenim oglasima na javnim površinama, gdje svoje usluge podučavanja
najčešće nude uspješniji studenti i (umirovljeni) nastavnici kako bi zaradili
dodatni prihod. U domaćem se medijskom prostoru povremeno izvještava da su
tutori stoga i česta meta Državnog inspektorata zbog rada „na crno” jer mnogi
od njih nisu registrirani kao obrtnici, a s obzirom na popularnost ovakvog
oblika dodatne zarade, nadzori su učestali. Razvijena aplikacija zasad ipak ne
zalazi u ovu pravnu i poreznu problematiku, već samo ističe cijene usluga te
plaćanje usluga ostaje između tutora i učenika izvan same aplikacije.

Praktični dio ovog rada, aplikacija \textit{repeticije-hr}, nastaje upravo
unutar prethodno predstavljenog konteksta, kao pokušaj da se postojeće, često
netransparentno i nejednako tržište privatnih instrukcija učini pravednijim,
preglednijim i dostupnijim većem broju učenika i studenata. Temeljna ideja
aplikacije je digitalno povezati učenike i tutore na jednom mjestu, uz jasan
uvid u kvalifikacije, ponudu predmeta, cijene, termine, lokaciju (uživo/\textit{online})
i povratne informacije drugih korisnika, čime se smanjuje ovisnost o „vezi“ i
uskim krugovima preporuka. Time se barem djelomično adresiraju problemi koje
ističu Jokić i Ristić Dedić {--} neravnomjerna dostupnost instrukcija,
netransparentnost ponude i snažna socijalna selektivnost {--} jer učenici iz
različitih dijelova Hrvatske, neovisno o tome žive li u većim centrima ili
manjim sredinama, dobivaju strukturiran digitalni prostor u kojem mogu lakše
pronaći podršku koja im je potrebna.

U poglavljima koja slijede prvo će kratko biti predstavljene korištene
tehnologije pri izradi aplikacije i objašnjena teoretska podloga, a zatim će se
detaljno i pregledno opisati poslovna logika i način na koji je sama aplikacija
implementirana.
