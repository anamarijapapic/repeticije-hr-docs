\section{Zaključak}

U ovom diplomskom radu predstavljen je razvoj web aplikacije za povezivanje
studenata i tutora koja pokriva proces od formiranja ponuda i rezervacije
termina do održavanja repeticija, uz implementiranu podršku za online video
pozive. Sustav je podijeljen na prezentacijski, aplikacijski, podatkovni i
komunikacijski sloj, pri čemu odabrane tehnologije omogućuju skalabilnu i
modularnu arhitekturu.

Posebna pažnja posvećena je sigurnosti i pouzdanosti: implementirani su
autentikacija i autorizacija s ulogama studenta, tutora i administratora,
impersonacija radi lakšeg otklanjanja problema, zaštita API ruta preko
\textit{proxy} i \textit{middleware} sloja, validacija vremenske zone te
integracija Google Maps API-ja i automatiziranih \textit{cron} zadataka. Uz to,
podrška za online repeticije putem WebRTC-a, sustav obavijesti putem e-pošte i
filtriranje ponuda prema lokaciji i dostupnosti tutora čine aplikaciju
praktičnim alatom za organizaciju repeticija.

Odabrani dizajn omogućuje daljnje proširenje funkcionalnosti (primjerice
integracija plaćanja, internacionalizacija i sl.), ali je u prvoj verziji
sustava svjesno primijenjen YAGNI princip kako bi rješenje ostalo razumljivo,
održivo i fokusirano na ključne potrebe korisnika na lokalnom, hrvatskom
tržištu.
