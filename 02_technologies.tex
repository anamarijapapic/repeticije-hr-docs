\section{Tehnologije i teoretska podloga}

\subsection{Next.js, React, TypeScript, Node.js, \texttt{npm}}

\textbf{Next.js} je razvojni okvir (engl.\ \textit{framework}) otvorenog k\^oda za React
koji omogućava objedinjeni (engl.\ \textit{full-stack}) razvoj web aplikacija,
što znači da obuhvaća razvoj i klijentskog (engl.\ \textit{front-end}) i
poslužiteljskog (engl.\ \textit{back-end}) dijela aplikacije. Razvijen je od
strane tvrtke Vercel, a izvorni k\^od dostupan je na platformi
GitHub~\cite{nextJsGitHub} gdje je prva verzija objavljena 2016.\ godine.

Temeljen je na JavaScriptu i izvršava se u okruženju \textbf{Node.js}, pri čemu
Node.js omogućava pokretanje JavaScript k\^oda izvan preglednika, na
poslužitelju, što je preduvjet za poslužiteljsko renderiranje i izgradnju
aplikacije~\cite{nodeJsAbout}. U praksi se Next.js projekti gotovo uvijek
razvijaju uz \textbf{TypeScript}, nadskup JavaScripta razvijen od strane
Microsofta, koji uvodi statičku tipizaciju i provjeru k\^oda u vrijeme
prevođenja, čime se smanjuje broj pogrešaka u složenijim
aplikacijama~\cite{typescriptDocs}.

Upravljanje ovisnostima obavlja se putem \textbf{\texttt{npm}}-a (engl.\
\textit{Node Package Manager}) ili alternativno
\texttt{pnpm}/\texttt{yarn}/\texttt{bun} upravitelja paketa, koji omogućuju
jednostavnu instalaciju, ažuriranje i skriptiranje razvojnih zadataka u Node.js
okruženju. Paketi (biblioteke) se preuzimaju iz javnog repozitorija
\href{https://www.npmjs.com/}{npmjs.com}, a definiraju se u datoteci
\texttt{package.json} unutar projekta, a pri instalaciji se kreira i datoteka
\texttt{package-lock.json} koja zaključava točne verzije ovisnosti radi
konzistentnog okruženja. U direktoriju \texttt{node\_modules} pohranjuju se sve
instalirane ovisnosti projekta~\cite{npmDocs}.

\textbf{React} (React.js) je slobodan softver otvorenog k\^oda tzv. FOSS (engl.\
\textit{Free and Open Source Software}). Radi se o JavaScript biblioteci za
izgradnju korisničkih sučelja, razvijenoj od strane Facebooka (sada Meta).
Programski k\^od javno je objavljen 2013.\ godine, a od tada je prigrljen od
strane zajednice i pozicionirao se kao jedna od najpopularnijih biblioteka za
razvoj web aplikacija~\cite{reactGitHub}.

React se često opisuje kroz tri osnovne značajke: \textit{declarative},
\textit{component-based} i sloganom \textit{„learn once, write anywhere”}.
Deklarativan pristup znači da razvojni programer opisuje kakvo korisničko
sučelje želi u određenom stanju aplikacije, a React se brine za učinkovit
prikaz i osvježavanje DOM-a (engl.\ \textit{Document Object Model}), što
olakšava razumijevanje i održavanje k\^oda. Komponentni pristup podrazumijeva
da se sučelje sastoji od malih, ponovno iskoristivih komponenti koje
enkapsuliraju logiku i prikaz pa se iste komponente mogu koristiti na više
mjesta i u različitim projektima. Načelo \textit{„learn once, write anywhere”}
odnosi se na činjenicu da se ista znanja i koncepti mogu primijeniti u web
aplikacijama, ali i u izradi mobilnih (React Native) i desktop aplikacija, bez
ponovnog učenja potpuno novog programskog modela~\cite{reactLegacy}.

React se temelji na konceptu komponenti, pri čemu se u povijesnom razvoju
razlikuju klasne (engl.\ \textit{class}) komponente i funkcijske (engl.\
\textit{function}) komponente, koje su trenutno preporučeni način pisanja
komponenti. Uvođenjem kuka (engl.\ \textit{hooks}) preko Hook API-ja (engl.\
\textit{Application Programming Interface}) funkcijske komponente dobile su
mogućnost upravljanja stanjem i \textit{side} efektima bez potrebe za klasama,
čime se dodatno pojednostavilo strukturiranje logike i ponovna iskoristivost
k\^oda. React koristi virtualni DOM {--} laganu, u memoriji pohranjenu
reprezentaciju strukture korisničkog sučelja kako bi učinkovito uspoređivao
prethodno i novo stanje te primjenjivao samo nužne promjene u stvarnom DOM-u,
što značajno poboljšava performanse složenijih aplikacija. JSX (engl.\
\textit{JavaScript XML}) je sintaksno proširenje za JavaScript koji omogućuje
zapis komponenti u obliku HTML-u slične sintakse, pri čemu se JSX u pozadini
prevodi u pozive funkcije \texttt{React.createElement}, a razvojnom programeru
olakšava čitanje i strukturiranje k\^oda~\cite{reactReference}. U kombinaciji s
TypeScriptom koristi se proširenje TSX (engl.\ \textit{TypeScript JSX}). U
novijim verzijama React uvodi i \textit{Server Components} {--} komponente koje
se izvršavaju isključivo na poslužitelju, generiraju HTML bez slanja JavaScript
k\^oda na klijent i tako smanjuju veličinu paketa te povećavaju sigurnost i
performanse, osobito u kombinaciji s razvojnim okvirima poput Next.js-a.

Next.js nadograđuje React dodajući brojne značajke i optimizacije koje
olakšavaju izgradnju skalabilnih i visokoučinkovitih web aplikacija. Uvodi
višestruke načine renderiranja stranica i napredne optimizacije koje su posebno
važne za performanse i SEO.\@ Osim klasičnog renderiranja na strani klijenta
    {--} CSR (engl.\ \textit{Client-Side Rendering}), Next.js podržava
poslužiteljsko renderiranje {--} SSR (engl.\ \textit{Server-Side Rendering}),
statičko generiranje stranica {--} SSG (engl.\ \textit{Static Site Generation})
te inkrementalno statičko obnavljanje {--} ISR (engl.\ \textit{Incremental
    Static Regeneration}), pri čemu se većina sadržaja isporučuje kao statički
HTML, a pojedine stranice se povremeno regeneriraju kada se podaci promijene.
Next.js također automatski dijeli k\^od u manje dijelove (engl.\ \textit{code
    splitting}) i koristi usmjeravanje (engl.\ \textit{routing}) bazirano na
strukturi direktorija, poznato kao \textit{file-based routing}. To znači da se
samo nužni dijelovi k\^oda učitavaju za svaku stranicu, čime se smanjuje
vrijeme učitavanja i poboljšava korisničko iskustvo. U novijim verzijama uveden
je App Router s podrškom za React Server Components, strujanje (engl.\
\textit{streaming}) sadržaja i Server Actions, što omogućuje da se dio logike
obrade podataka izvršava na poslužitelju uz manju količinu potrebnog JavaScript
k\^oda na klijentu, što pruža brži prikaz sadržaja korisniku~\cite{nextJsDocs}.

Osim spomenutih značajki, Next.js sadrži integrirane alate za optimizaciju
slika automatski generirajući različite veličine, koristi lijeno učitavanje
(engl.\ \textit{lazy loading}) i moderne formate slika, čime se poboljšavaju
Core Web Vitals metrike i ukupne performanse web aplikacija. U kombinaciji s
podrškom za TypeScript \textit{„out-of-the-box”}, integracijom s Vercelom za
jednostavnu isporuku te bogatim ekosustavom dodataka, Next.js se pozicionira
kao cjelovit razvojni okvir za izradu skalabilnih, produkcijski spremnih
aplikacija~\cite{nextJsDocs}.

U trenutku izrade rada najnovija stabilna verzija Next.js-a je 16, dok je React
dosegao verziju 19 te su iste korištene u izradi praktičnog rada, uz TypeScript
verzije 5.x. Korištena verzija Node.js-a je 24.x LTS (engl.\ \textit{Long Term
    Support}) uz \texttt{npm} verzije 11.x.

\subsection{UI sloj: \texttt{shadcn/ui}, Radix UI, Tailwind CSS}

Sloj korisničkog sučelja (engl.\ UI {--} \textit{User Interface}) aplikacije
temelji se na kombinaciji biblioteka \texttt{shadcn/ui}, Radix UI i Tailwind
CSS, koje su usko povezane i nadograđuju se jedna na drugu.

\textbf{\texttt{shadcn/ui}} je zbirka unaprijed pripremljenih React komponenti (npr.\ avatar,
botuni, bedževi, kartice za prikaz sadržaja, dijaloški okviri, padajući
izbornici, polja i grupe za unos, navigacijski izbornici, tablice itd.) koje su
izgrađene na Radix UI „primitivima” i stilizirane pomoću Tailwind CSS-a.
Umjesto da se isporučuje kao klasična biblioteka ovisnosti, \texttt{shadcn/ui}
generira stvarne izvorišne datoteke komponenti koje se kopiraju u projekt, čime
razvojni programer zadržava potpunu kontrolu nad k\^odom, može ga prilagoditi
specifičnim potrebama aplikacije i izbjegava dodatni teret zasebne UI
biblioteke dok se aplikacija izvršava (engl.\ \textit{runtime}). Ovo je osobito
korisno u većim projektima, gdje se očekuju dugoročno održavanje, prilagodba
dizajna i dosljedan vizualni identitet~\cite{shadcnDocs}.

\textbf{Radix UI} predstavlja skup pristupačnih (engl.\ \textit{accessible})
niskorazinskih komponenti tzv. „primitiva” poput dijaloških okvira, skočnih
prozora, kartica (engl.\ \textit{tabs}), skočnih izbornika i slično. Fokus
Radix UI-ja je na ispravnom ponašanju i pristupačnosti: osigurava ispravne ARIA
(engl.\ \textit{Accessible Rich Internet Applications}) atribute, upravljanje
fokusom i podršku za tipkovničku navigaciju u skladu s WAI-ARIA (engl.\
\textit{Web Accessibility Initiative {--} ARIA}) smjernicama, dok izgled i
stilizacija ostaju u potpunosti u domeni razvojnih timova~\cite{radixUiPrimitives}.

\textbf{Tailwind CSS} je utilitaristički (engl.\ \textit{utility-first}) razvojni okvir
za CSS koji umjesto gotovih vizualnih komponenti nudi velik broj malih,
jednoznačnih klasa za oblikovanje (npr.\ razmaci, boje, tipografija, raspored {--}
fleksibilni i mrežni (engl.\ \textit{grid}) rasporedi). Te se klase kombiniraju
izravno u markup-u ili JSX/TSX k\^odu, čime se smanjuje potreba za pisanjem
zasebnih CSS datoteka i olakšava održavanje dosljednog dizajna u komponentnom
pristupu Reacta. Tailwind je visokokonfigurabilan putem konfiguracijske
datoteke koja definira dizajnerske tokene (boje, tipografiju, radijuse,
razmake), a u produkcijskom okruženju alat automatski uklanja neiskorištene klase
(engl.\ \textit{tree shaking}), što rezultira malim konačnim CSS paketom i
boljim performansama~\cite{tailwindDocsCoreConceptsStylingWithUtilityClasses}.

U razvijenoj aplikaciji kombinacija predstavljenih UI biblioteka omogućava
izradu modernog, responzivnog i pristupačnog korisničkog sučelja uz dobru
ravnotežu između brzine razvoja, kontrole nad dizajnom i tehničke kvalitete.

\subsection{Baza podataka: PostgreSQL, Prisma ORM, Supabase}

\textbf{PostgreSQL}, \textbf{Prisma ORM}, \textbf{Supabase} (CLI, lokalni Docker razvoj, hosting, real-time funkcionalnosti)

\subsection{WebRTC}

\subsection{Verzioniranje k\^oda, hosting i deployment, email servis}

\textbf{Git}, \textbf{GitHub}, \textbf{Vercel}, \textbf{Resend}
